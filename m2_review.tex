\documentclass{article}
\usepackage[utf8]{inputenc}
\usepackage{enumerate}
\usepackage{amsmath}
\usepackage{amssymb}
\usepackage{amsfonts}
\usepackage{amstext}
\usepackage{amsthm}
\usepackage{mathtools}
\DeclarePairedDelimiter{\ceil}{\lceil}{\rceil}

\title{Calculus III Midterm 2 Practice}
\author{Tyler Franklin}

\begin{document}
\maketitle

\section{Surface Area of a Sphere $x^2 + y^2 + z^2 = 1^2$ }
\begin{enumerate}[a.]
	\item \textbf{Cylindrically -- cylindrical bands on the surface }

            Recall cylindrical Jacobian determinant $ r $ and use the gradient form (assume $R = 1$):
            \[ A = \iint_{D}\sqrt{ (\frac{\partial z}{\partial x})^2 + (\frac{\partial z}{\partial y})^2 + R^2 } \cdot dA \]
            Solve for z:
            \[ z = \pm\sqrt{1-x^{2}-y^{2}} \]
            Compute partials with respect to x then y:
            \[ \frac{\partial z}{\partial x} = \frac{\partial \:}{\partial \:u}\left(\sqrt{u}\right)\frac{\partial \:}{\partial \:x}\left(1-x^2-y^2\right) = \frac{1}{2\sqrt{u}}\left(-2x\right) = -\frac{x}{\sqrt{1-x^2-y^2}} \]
            \[ \frac{\partial z}{\partial y} = \frac{\partial \:}{\partial \:u}\left(\sqrt{u}\right)\frac{\partial \:}{\partial \:y}\left(1-x^2-y^2\right) = \frac{1}{2\sqrt{u}}\left(-2y\right) = -\frac{y}{\sqrt{1-x^2-y^2}} \]
            Plug values and simplify:
            \[ A(s) = \iint_{D}\sqrt{\left(-\frac{x}{\sqrt{1-x^2-y^2}}\right)^2+\left(-\frac{y}{\sqrt{1-x^2-y^2}}\right)^2+1}  \cdot dA \]
            \[ = \iint_{D}\sqrt{\frac{x^2+y^2}{1-x^2-y^2}+1}  \cdot dA \]
            \[ = \iint_{D}\sqrt{\frac{1}{1-(x^2+y^2)}}  \cdot dA \]
            Convert to cylindrical (recall $x^2 + y^2 = r^2$):
            \[ = \iint_{D}\sqrt{\frac{1}{1-r^2}}  \cdot dA \]
            Plug values (recall $0 \leq \theta \leq 2\pi $ and $0 \leq r \leq R $, also multiply integral by 2 to account for both hemispheres) and calculate:
            \[ 2\int_{0}^{1}\int_{\theta=0}^{2\pi} \frac{1}{\sqrt{1-r^2}}\cdot r \cdot dr \cdot d\theta = 2\cdot \int _0^1\frac{2\pi r}{\sqrt{-r^2+1}}dr = 4\pi \]

	\item \textbf{Spherically -- rectangles on surface}

            Recall the Cartesian $\rightarrow$ spherical conversions:
            \[ x = \rho\sin(\phi)\cos(\theta)   \]
            \[ y = \rho\sin(\phi)\sin(\theta)  \]
            \[ z = \rho\cos(\phi)  \]
            ...that any rectangle on the surface will be described as:
            \[ (\rho\cdot\sin(\phi)\cdot d \theta)\cdot(\rho\cdot d \phi) = \rho^2\cdot\sin(\phi)\cdot d\theta\cdot d\phi \]
            ...and that the coordinate values will range as follows:
            \[ 0 \leq \phi \leq \pi  \]
            \[ 0 \leq \theta \leq 2\pi  \]
            \[ 0 \leq \rho \leq R \]
            ...and our given value of R is 1 ($\rho = 1$).

            With this information we can plug and calculate:
            \[ A =\int_{0}^{\pi}\int_{0}^{2\pi} \rho^2\cdot\sin(\phi)\cdot d\theta\cdot d\phi  \]
            \[ \int_{0}^{\pi}\int_{0}^{2\pi} (1)^2\cdot\sin(\phi)\cdot d\theta\cdot d\phi\]
            \[ \int_{0}^{\pi}2\pi \cdot \sin(\phi)\cdot d\phi \]
            \[ 2\pi \cdot (-\cos(\pi) + \cos(0)) \]
            \[ 2\pi \cdot (-(-1)+(1)) \]
            \[ 4\pi\]

	\item \textbf{Stereographically -- vector intersections with the surface}
	\[ A = \iint_{D} |r_u \times r_v| \cdot dA \]
	\[x(u,v)=\frac{2u}{\:1^2+u^2+v^2}\]
	\[y(u,v)=\frac{2v}{\:1^2+u^2+v^2}\]
	\[z(u,v)=\frac{-1^2+u^2+v^2}{\:1^2+u^2+v^2}\]
	\[\frac{\partial x}{\partial u} = \frac{2\left(-u^2+v^2+1\right)}{\left(1+u^2+v^2\right)^2}\]
	\[\frac{\partial y}{\partial u} = -\frac{4vu}{\left(1+u^2+v^2\right)^2}\]
	\[\frac{\partial z}{\partial u} = \frac{4u}{\left(1+u^2+v^2\right)^2}\]
	\[\frac{\partial x}{\partial v} = -\frac{4uv}{\left(1+u^2+v^2\right)^2}\]
	\[\frac{\partial y}{\partial v} = \frac{2\left(1+u^2-v^2\right)}{\left(1+u^2+v^2\right)^2}\]
	\[\frac{\partial z}{\partial v} = \frac{4v}{\left(1+u^2+v^2\right)^2}\]
	\[r_u := \frac{\partial x}{\partial u}i + \frac{\partial y}{\partial u}j + \frac{\partial z}{\partial u}k = \left(\frac{2\left(-u^2+v^2+1\right)}{\left(1+u^2+v^2\right)^2}\right)i + \left(-\frac{4vu}{\left(1+u^2+v^2\right)^2}\right)j + \left(\frac{4u}{\left(1+u^2+v^2\right)^2}\right)k\]
	\[r_v := \frac{\partial x}{\partial v}i + \frac{\partial y}{\partial v}j + \frac{\partial z}{\partial v}k = \left(-\frac{4uv}{\left(1+u^2+v^2\right)^2}\right)i + \left(\frac{2\left(1+u^2-v^2\right)}{\left(1+u^2+v^2\right)^2}\right)j + \left(\frac{4v}{\left(1+u^2+v^2\right)^2}\right)k\]
	\[ A = \iint_{D} |r_u \times r_v| \cdot dA \]
	\[\iint_{D} \left|\begin{pmatrix}\frac{8u^3-24uv^2+8u}{\left(u^2+v^2+1\right)^4}&-\frac{8v}{\left(u^2+v^2+1\right)^3}&\frac{4u^4-24u^2v^2+4v^4-4}{\left(u^2+v^2+1\right)^4}\end{pmatrix}\right| dA\]
	\[\left|\iint_{D} \frac{8u^3-24uv^2+8u}{\left(u^2+v^2+1\right)^4} \cdot dA \cdot i + \iint_{D}\frac{4u^4-24u^2v^2+4v^4-4}{\left(u^2+v^2+1\right)^4} \cdot dA \cdot j + \iint_{D}\frac{8v}{\left(u^2+v^2+1\right)^3}\cdot dA \cdot k \right|\]
	\[\int_{-\infty}^{\infty}\int_{-\infty}^{\infty} \frac{8u^3-24uv^2+8u}{\left(u^2+v^2+1\right)^4} \cdot du \cdot dv = 0\]
	\[\int_{-\infty}^{\infty}\int_{-\infty}^{\infty}\frac{4u^4-24u^2v^2+4v^4-4}{\left(u^2+v^2+1\right)^4} \cdot  du \cdot dv = -i\pi\cdot\int_{\infty}^{-\infty}\left(-1\:-\:v^2\right)^{-\frac{5}{2}}\cdot dv = -i\pi \cdot \frac{4i}{3} = \frac{4\pi}{3}\]
	\[\int_{-\infty}^{\infty}\int_{-\infty}^{\infty}\frac{8v}{\left(u^2+v^2+1\right)^3} \cdot  du \cdot dv = \int_{-\infty}^{\infty}\frac{3\pi v\sqrt{v^2+1}}{v^6+3v^4+3v^2+1} \cdot dv = 0\]
	\[\left|(0)i + \left(\frac{4\pi}{3}\right)j+(0)k\right|=\sqrt{0^2+\left(\frac{4\pi }{3}\right)^2+0^2}=\frac{4\pi}{3}\]

\end{enumerate}

\section{Volume of a Sphere $x^2 + y^2 + z^2 = 1^2$}
\begin{enumerate}[a.]
	\item \textbf{Cylindrically -- cylindrical "discs" to form sphere}

	    Convert $x^2+y^2$ to $r$ and solve for $r$:
	    \[ r=\sqrt{1-z^2}\]
        Recall that our cylindrical coordinates range as follows:
        \[0 \leq r \leq \sqrt{1-z^2}\]
        \[0 \leq \theta \leq 2\pi \]
        \[-R \leq z \leq R \]
        Since the negative  bound turns into two absolute integrals anyway, we can save ourselves the trouble and just range $0 \leq z \leq R$ and multiply the result by 2 to account for both hemispheres.

        Finally, we plug these values and calculate. To get concentric "discs" stacking from the equator to the pole of the hemisphere, we must integrate with respect to $z$ \textbf{last}:
        \[ V = 2\cdot \int _0^1\int _0^{2\pi }\int _0^{\sqrt{1-z^2}}rdrd\theta dz \]
        \[ =2\cdot \int _0^1\int _0^{2\pi }\frac{1-z^2}{2}d\theta dz \]
        \[=2\cdot \int _0^1\pi \left(1-z^2\right)dz\]
        \[=2\pi\codt (\int _0^11dz-\int _0^1z^2dz)\]
        \[=2\pi\left(1-\frac{1}{3}\right)\]
        \[=\frac{4\pi }{3} \]

	\item \textbf{Spherically -- convert from Cartesian and calculate}

        Recall spherical Jacobian determinant $ \rho^2\sin(\phi) $ and the Cartesian $\rightarrow$ spherical conversions:
        \[ x = \rho\sin(\phi)\cos(\theta)   \]
        \[ y = \rho\sin(\phi)\sin(\theta)  \]
        \[ z = \rho\cos(\phi)  \]
        \[ x^2+y^2+z^2 = \rho^2 \]
        Our volume would take the form
        \[ V = \iiint_{R}dV \]
        \[ = \iiint_{R}\rho^2\sin(\phi) \cdot d\rho d\theta d\phi \]
        Recall that the spherical coordinate values will range as follows:
        \[ 0 \leq \phi \leq \pi  \]
        \[ 0 \leq \theta \leq 2\pi  \]
        \[ 0 \leq \rho \leq R \]
        Plug these values and calculate:
        \[ V = \int_{0}^{\pi}\int_{0}^{2\pi}\int_{0}^{(1)}\rho^2\sin(\phi) \cdot d\rho d\theta d\phi \]
        \[ = \int_{0}^{\pi}\int_{0}^{2\pi}\frac{1}{3}\sin(\phi) d\theta d\phi \]
        \[ = \frac{1}{3}\int_{0}^{\pi}2\pi\sin(\phi) d\phi \]
        \[ = \frac{2\pi}{3}\left[-\cos \left(\phi\right)\right]^{\pi }_0\]
        \[ = \frac{2\pi}{3}(1-\left(-1\right))\]
        \[ = \frac{4\pi}{3} \]

	\item \textbf{Cylindrically -- convert from Cartesian and calculate}

	    As before, solve for z and convert to $r$ immediately:
	    \[ z=\pm\sqrt{1-(x^2+y^2)} = \pm\sqrt{1-r^2}\]
        Recall cylindrical Jacobian determinant $ r $ and that cylindrical coordinates range (for one hemisphere, which will multiply later):
        \[0 \leq \theta \leq 2\pi \]
        \[0 \leq r \leq R \]
        \[0 \leq z \leq +\sqrt{1-r^2}\]
        Since the negative z bound turns into two absolute integrals anyway, we can save ourselves the trouble and just range $0 \leq z \leq +\sqrt{1-r^2}$ and multiply the result by 2 to account for both hemispheres.

        Finally, we plug these values and calculate. To get concentric cylinders spreading from the center to the outside of the hemisphere, we must integrate with respect to $r$ \textbf{last}:
        \[ V = \int_{r = 0}^{r = (1)} \int_{\theta = 0}^{\theta = 2\pi} 2\cdot\int_{z = 0}^{z = \sqrt{1-r^2}} r \cdot dz \cdot d\theta \cdot dr \]
        \[ V = \int_{r = 0}^{r = (1)} \int_{\theta = 0}^{\theta = 2\pi} 2\cdot r \cdot \sqrt{1-r^2} \cdot d\theta \cdot dr \]
        \[=\int _0^14\pi r\sqrt{-r^2+1} \cdot dr\]
        \[=4\pi \cdot \int _1^0-u^2 \cdot du\]
        \[=4\pi \left(-\left(-\int _0^1u^2 \cdot du\right)\right)\]
        \[=4\pi \left[\frac{u^3}{3}\right]^1_0\]
        \[=\frac{4\pi }{3} \]

	\item \textbf{Spherically -- order integral to form spherical "shells" emanating from origin}

        Our volume would take the form
        \[ V = \iiint_{R}dV \]
        ...with integration with respect to $\rho$ occurring \textbf{last} in order to get the concentric "shells" effect:
        \[ = \iiint_{R}\rho^2\sin(\phi) \cdot d\phi d\theta d\rho \]
        Recall that the spherical coordinate values will range as follows:
        \[ 0 \leq \phi \leq \pi  \]
        \[ 0 \leq \theta \leq 2\pi  \]
        \[ 0 \leq \rho \leq R \]
        Plug these values and calculate:
        \[ V = \int_{0}^{(1)}\int_{0}^{2\pi}\int_{0}^{\pi}\rho^2\sin(\phi) \cdot d\phi d\theta d\rho \]
        \[ \int_{0}^{(1)}\rho^2\int_{0}^{2\pi} \left[-\cos \left(\phi\right)\right]_0^{\pi } d\theta d\rho \]
        \[ \int_{0}^{(1)}\rho^2\int_{0}^{2\pi} (1-\left(-1\right)) d\theta d\rho \]
        \[ 2 \cdot \int_{0}^{(1)} 2\pi \cdot \rho^2 d\rho \]
        \[ 4\pi \cdot \frac{1}{3} \]
        \[ = \frac{4\pi}{3} \]

	\item \textbf{Stereographically -- vector intersections with the surface}
\end{enumerate}

\section{Gaussian Normal Curve as a Probability Density Function: Prove the volume is 2$\pi$ using polar coordinates}
    First, observe that:
    \[ \int_{-\infty}^{\infty} e^-\frac{x^2}{2} dx = \int_{-\infty}^{\infty} e^-\frac{y^2}{2} dy\]
    Which implies:
    \[ \int_{-\infty}^{\infty} e^-\frac{x^2}{2} dx \cdot \int_{-\infty}^{\infty} e^-\frac{y^2}{2} dy \]
    \[ = \int_{-\infty}^{\infty} \int_{-\infty}^{\infty} e^-\frac{-(x^2+y^2)}{2} dx dy \]
    Now we can can convert to polar coordinates:
    \[ \int_{0}^{\infty} \int_{0}^{2\pi} e^{-\frac{r^2}{2}} r \cdot d\theta dr \]
    And calculate:
    \[ 2\pi \cdot \int_{0}^{\infty} e^{-\frac{r^2}{2}} r \cdot d\theta dr\]
    \[ 2\pi \cdot (\lim _{\theta \to \infty \:}\left(-e^{-\frac{\theta ^2}{2}}\right) - \lim _{\theta \to \:0+}\left(-e^{-\frac{\theta ^2}{2}}\right))\]
    \[ 2\pi \cdot ((0)-\left(-1\right))\]
    \[ 2\pi\]

\section{Sequences and Series}

\begin{enumerate}[a.]
	\item What does it mean for a sequence to converge?

	When a sequence converges, that means that as you get further and further along the sequence, the terms get closer and closer to a specific limit (usually a real number). For example, to see if the sequence $\frac{1}{n^2}$ converges, we may observe that:
	\[\lim _{n \to \infty \:}\frac{1}{n^2}=0\]

	\item What does it mean for a series to converge?

    A series is the sum of a sequence. When a series converges, the sums get closer and closer to a specific limit as we add more and more terms up to infinity. For example, Zeno's dichotomy paradox (halving the distance between objects repeatedly results in them constantly moving but "never" quite reaching each other) is a series:
    \[D = 1 - \sum_{n=0}^{\infty}\frac{1}{2^n}\]

	\item What does it mean for a series to converge absolutely?

    A series $\sum a_n$ is called absolutely convergent if $\sum |a_n|$ is convergent. If $\sum a_n$ is convergent and $\sum |a_n|$ is divergent we call the series conditionally convergent.

	\item If a series converges absolutely, does it converge? Prove it.

    First notice that $|a_n|$ is either $a_n$ or it is $-a_n$ depending on its sign. This means that we can then say, $0 \leq a_n + |a_n| \leq 2|a_n|$

    Now, since we are assuming that $\sum |a_n| $ is convergent then $\sum |a_n| $ is also convergent since we can just factor the 2 out of the series and 2 times a finite value will still be finite. This however allows us to use the Comparison Test to say that $\sum(a_n + |a_n|)$ is also a convergent series.

    Finally, we can write, $\sum a_n = \sum(a_n + |a_n|) - \sum |a_n|$ and so $\sum a_n$ is the difference of two convergent series and so is also convergent.

	\item If a series converges, does it converge absolutely? Provide counterexample.

	Not necessarily, a counterexample would be an alternating series such as
	\[\sum_{n=0}^{\infty}\frac{(-1)^n}{n}\]
	which converges but not absolutely, since
	\[\sum_{n=0}^{\infty}|\frac{(-1)^n}{n}| = \sum_{n=0}^{\infty}\frac{1}{n}\]
	also known as the harmonic series, which does not converge.
	\item What is a geometric sequence?
	Geometric sequences follow a pattern of \textbf{multiplying a fixed amount (not zero) from one term to the next}. The number being multiplied each time is constant.
	\item What is a geometric series?
	A geometric series is the sum of the terms in a geometric sequence. If the sequence has a definite number of terms, the simple formula for the sum is
    \[s(n) = \frac{a_1(1-r^n)}{1-r}\]
    This form of the formula is used when the number of terms ($n$), the first term ($a_1$), and the common ratio ($r$) are known.
	\item What criteria guarantee convergence of either a geometric sequence or geometric series?
	When the absolute of the ratio $r$ of is less than one:
	\[|r| < 1\]
	\item Prove that the harmonic series diverges, while the series $s(n) = \frac{1}{n^2} $ converges
	\[s(n) = \frac{1}{n} = 1+\frac{1}{2}+\frac{1}{3}+\frac{1}{4}+\frac{1}{5}+\frac{1}{6}\cdots\]
	\[\geq 1+\frac{1}{2}+(\frac{1}{4}+\frac{1}{4})+(\frac{1}{8}+\frac{1}{8}+\frac{1}{8}+\frac{1}{8})\cdots\]
	\[1+\frac{1}{2}+\frac{1}{2}+\frac{1}{2}+\frac{1}{2}+\frac{1}{2}+\frac{1}{2}+\frac{1}{2}\cdots\]
	With this method, we can use the comparison theorem to prove that:
	\[\iff \sum_{n=1}{2^k}\frac{1}{n}\geq 1+\frac{k}{2}\]
	Since $1+\frac{k}{2}$ is clearly infinity, $\sum_{n=1}{\infty}\frac{1}{n}$ must be as well.
	While it might take more work to calculate what $\frac{1}{n^2}$ converges to exactly, we can prove it does converge by using the integral test:
	\[\sum_{n=2}^{\infty}\frac{1}{n^2}<\int_{1}^{\infty}\frac{1}{x^2}dx<\sum_{n=1}^{\infty}\frac{1}{n^2}\]
	The first sum is a rectangular underestimate for the integral, and the second sum is an overestimate. Thus, our series sums to almost the integral, the error being at most the value of the first term.
	\[\int_{1}^{\infty}\frac{1}{x^2}dx = \lim _{x\to \infty \:}\left(-\frac{1}{x}\right)- \lim _{x\to \:1+}\left(-\frac{1}{x}\right) = (0)-(-1) = 1\]
	This tells us that the sequence converges and it converges to something between 1 and 2.
	\item Prove with the ratio test that series $e^z$ is absolutely convergent.
	First, we know that
	\[e^z=\sum_{n=0}^{\infty}\frac{x^n}{n!}\]
	Using the ratio test we can show that it converges:
	\[a_n=\frac{z^n}{n!} \]
	\[a_{n+1}=\frac{z^{n+1}}{(n+1)!}\]
	\[\frac{a_{n+1}}{a_n}=\frac{\frac{z^{n+1}}{(n+1)!}}{\frac{z^n}{n!}} = \frac{z^{n+1}n!}{\left(n+1\right)!z^n}=\frac{z^{n-n+1}n!}{\left(n+1\right)!}=\frac{zn!}{\left(n+1\right)!}\]
	Note the rule:
	\[\frac{n!}{\left(n+m\right)!}=\frac{1}{\left(n+1\right)\cdot \left(n+2\right)\cdots \left(n+m\right)} \Rightarrow \frac{zn!}{\left(n+1\right)!}=\frac{z}{n+1}\]
	Finally, we can show that
	\[\lim_{n\to\infty}|\frac{z}{n+1}|=|z|\cdot(0)=0  \; (\forall \;  z \in \mathbb{R})\]
	\item Prove using the corollary of the root test that the radius of convergence of $e^z$ is infinite.

	We proved above that $e^z$ converges for all $z\in\mathbb{R}$, however perhaps we can also do so using the root test. The root test tells us that if the following is true, our sequence converges:
	\[\lim_{n\to\infty}|s(n)|^\frac{1}{n} < 1 \]
	So we may proceed:
	\[\lim_{n\to\infty}|s(n)|^\frac{1}{n}\]
	\[\lim_{n\to\infty}|\frac{z^n}{n!}|^\frac{1}{n}\]
	\[|x^\infty|^{\frac{1}{\infty}}\cdot\lim_{n\to\infty}|\frac{1}{n!}|^\frac{1}{n}\]
	\[(1)\cdot\lim_{n\to\infty}\frac{1}{\infty!^\frac{1}{\infty}}\]
	\[\frac{1}{\infty!^\frac{1}{\infty}}=\frac{1}{\infty}=0\]
	Thus for all $z\in\mathbb{R}$ we find that $e^z$ runs to 0, which means the radius of convergence is infinite in both directions.

	\item Prove that Euler's formula is true.
	\[e^{i\theta} = cos(\theta) + i\sin(\theta)\]
    First, we define e, cos, and sin as power series:
    \[cos(\theta) = \sum_{n=0}^{\infty} \frac{(-1)^n}{2n!} \cdot (\theta)^2n = 1-\frac{\theta^2}{2!}+\frac{\theta^4}{4!}-\frac{\theta^6}{6!}+\dots\]
    \[sin(\theta) = \sum_{n=0}^{\infty} \frac{(-1)^{n+1}}{(2n+1)!} \cdot (\theta)^{2n+1} = \theta-\frac{\theta^3}{3!}+\frac{\theta^5}{5!}-\frac{\theta^7}{7!}+\dots\]
    \[e^{i\theta} = \sum_{n=0}^{\infty} \frac{1}{n!} \cdot (i\theta)^n = 1+\frac{i\theta}{1!}+\frac{i^2\theta^2}{2!}+\frac{i^3\theta^3}{3!}+\frac{i^4\theta^4}{4!}+\frac{i^5\theta^5}{5!}+\frac{i^6\theta^6}{6!}+\frac{i^7\theta^7}{7!}\dots\]
    \[1+\frac{i\theta}{1!}+\frac{(-1)\theta^2}{2!}+\frac{(-1)i\theta^3}{3!}+\frac{(-1)^2\theta^4}{4!}+\frac{(-1)^2i\theta^5}{5!}+\frac{(-1)^3\theta^6}{6!}+\frac{(-1)^3i\theta^7}{7!}\dots\]
    \[(1-\frac{\theta^2}{2!}+\frac{\theta^4}{4!}-\frac{\theta^6}{6!}\dots)+\frac{i\theta}{1!}-\frac{i\theta^3}{3!}+\frac{i\theta^5}{5!}-\frac{i\theta^7}{7!}\dots\]
    \[cos(\theta)+i(\frac{\theta}{1!}-\frac{\theta^3}{3!}+\frac{\theta^5}{5!}-\frac{\theta^7}{7!}\dots)\]
    \[\cos{\theta}+i\sin{\theta} \]
	\[\therefore \; e^{i\theta}=\cos{\theta}+i\sin{\theta} \]

\end{enumerate}

\section{Using series to calculate a pesky integral:}
    \[\int _{-1 }^{1 \:}e^{-\frac{x^2}{2}}dx\]
    Recall that
    \[e^z= \sum_{n=0}^{\infty}\frac{x^n}{n!}\]
    \[\Rightarrow \int _{-1 }^{1 \:}e^{-\frac{x^2}{2}}dx=\int _{-1 \:}^{1 \:\:}\sum_{n=0}^{\infty}\frac{(-\frac{x^2}{2})^n}{n!} dx \]

    Remember: a sequence $s(n)$ converges to a limit
    \[L \in\mathbb{R}\iff\forall \; \epsilon > 0  \; \exists \; N \in \mathbb{N} \; s.t. \; |s(n) - L| < \epsilon \; \forall \; n \in \mathbb{N} > N\]
    \[ \Rightarrow |\frac{2^{-n}}{n!} - (0)| < \epsilon\]
    While solving for $n$ in this case would be difficult, we can try some numbers to see how quickly the error value drops:
    \begin{center}
     \begin{tabular}{||c | c | c||}
     \hline
     $n$ & $s(n)$ & $\epsilon_n$ \\ [0.5ex]
     1 & $\frac{2^{-1}}{1!}$ & $\frac{1}{8}$ \\
     2 & $\frac{2^{-2}}{2!}$ & $\frac{1}{48}$ \\
     3 & $\frac{2^{-3}}{3!}$ & $\frac{1}{384}$ \\ [1ex]
     \hline
    \end{tabular}
    \end{center}
    It appears that in order to reach an error threshold of $2^-8 = \frac{1}{256}$ we will only need 4 terms.
    \[\int _{-1 \:}^{1 \:\:}\sum_{n=0}^{3}\frac{(-\frac{x^2}{2})^n}{n!} dx \]
    \[\int _{-1}^1\left|\frac{-x^{2\cdot \:0}}{2^0\cdot \:0!}+\frac{-x^{2\cdot \:1}}{2^1\cdot \:1!}+\frac{-x^{2\cdot \:2}}{2^2\cdot \:2!}+\frac{-x^{2\cdot \:3}}{2^3\cdot \:3!}\right|dx=\frac{669}{280}\]

\section{Using a series to calculate $\pi$}
    The Generalized binomial formula for $a\in \mathbb R\setminus \{\mathbb N\}$ says
    \[(1+x)^a= \sum_{k=0}^{\infty}{a\choose k} x^k\]
    Where: ${{a\choose k} =\frac{a(a-1)\cdots(a-k+1)}{k!}}$. In our case $a=\frac{1}{2}$ and $x=(-x^2)$:
    \[(1+(-x^2))^{(\frac{1}{2})}= \sum_{k=0}^{\infty}{(\frac{1}{2})\choose k} (-x^2)^k\]
    \[1+\left(\frac{1}{2}\right)\left(-x^2\right)+\frac{\left(\frac{1}{2}\right)\left(\left(\frac{1}{2}\right)-1\right)}{2!}\left(-x^2\right)^2+\frac{\left(\frac{1}{2}\right)\left(\left(\frac{1}{2}\right)-1\right)\left(\left(\frac{1}{2}\right)-2\right)}{3!}\left(-x^2\right)^3+\dots\]
    This can be simplified and expanded further:
    \[1-\frac{x^2}{2}-\frac{x^4}{8}-\frac{x^6}{16}-\frac{5x^8}{128}-\frac{7x^{10}}{256}-\frac{21x^{12}}{1024}\dots\]
    We can see above the following rather complicated $s(n)$:
    \[s(n) = \frac{\left(2n-3\right)!!}{2^nn!}\]
    While solving $\left|s(n) - L\right| < \epsilon$ for $n$ in this case would be difficult, we can try some numbers to see how quickly the error value drops:
    \begin{center}
     \begin{tabular}{|| c | c ||}
     \hline
     $n$ & $s(n)$ \\ [0.5ex]
     1 & $\frac{1}{2}$ \\
     2 & $\frac{1}{8}$ \\
     3 & $\frac{1}{16}$ \\
     4 & $\frac{5}{128}$ \\
     5 & $\frac{7}{256}$ \\
     6 & $\frac{21}{1024}$ \\
     7 & $\frac{33}{2048}$ \\
     [1ex]
     \hline
    \end{tabular}
    \end{center}
    It appears the answer to the question above is "not fast." After trying more numbers I found that an $N$ of $18$ dropped the coefficient within the threshold but I'm not sure how we would determine that quickly.

\section{Using series to calculate $\frac{\pi^2}{6}$ within an error of $2^{-8}$}
    Remember: a sequence $s(n)$ converges to a limit
    \[L \in\mathbb{R}\iff\forall \; \epsilon > 0  \; \exists \; N \in \mathbb{N} \; s.t. \; |s(n) - L| < \epsilon \; \forall \; n \in \mathbb{N} > N\]
    If we solve for n, this should give us the minimum term $N$ with regard to error %\epsilon$.
    \[ |\frac{1}{n^2} - (0)| < \epsilon\]
    \[n^2>\frac{1}{\epsilon}\]
    \[n>\sqrt{\frac{1}{\epsilon}}\]
    \[N=\ceil[\bigg]{\sqrt{\frac{1}{\epsilon}}}\]
    And finally, we can plug $2^{-8}$ to find $N$:
    \[N=\ceil[\bigg]{\sqrt{\frac{1}{(\frac{1}{256})}}}=\sqrt{256}=\sqrt{16^2}=16\]
    The \textbf{16th} term in the series should fall within the specified error.
\end{document}


 
\documentclass{article}
\usepackage[utf8]{inputenc}
\usepackage{enumerate}
\usepackage{amsmath}
\usepackage{amssymb}
\usepackage{amsfonts}
\usepackage{amstext}
\usepackage{amsthm}
\usepackage{mathtools}
\DeclarePairedDelimiter{\ceil}{\lceil}{\rceil}

\title{Calculus III Midterm 2 Practice}
\author{Tyler Franklin}

\begin{document}
\maketitle

\section{Surface Area of a Sphere $x^2 + y^2 + z^2 = 1^2$ }
\begin{enumerate}[a.]
	\item \textbf{Cylindrically -- cylindrical bands on the surface }
	      
	      Recall cylindrical Jacobian determinant $ r $ and use the gradient form (assume $R = 1$):
	      \[ A = \iint_{D}\sqrt{ (\frac{\partial z}{\partial x})^2 + (\frac{\partial z}{\partial y})^2 + R^2 } \cdot dA \]
	      Solve for z:
	      \[ z = \pm\sqrt{1-x^{2}-y^{2}} \]
	      Compute partials with respect to x then y:
	      \[ \frac{\partial z}{\partial x} = \frac{\partial \:}{\partial \:u}\left(\sqrt{u}\right)\frac{\partial \:}{\partial \:x}\left(1-x^2-y^2\right) = \frac{1}{2\sqrt{u}}\left(-2x\right) = -\frac{x}{\sqrt{1-x^2-y^2}} \]
	      \[ \frac{\partial z}{\partial y} = \frac{\partial \:}{\partial \:u}\left(\sqrt{u}\right)\frac{\partial \:}{\partial \:y}\left(1-x^2-y^2\right) = \frac{1}{2\sqrt{u}}\left(-2y\right) = -\frac{y}{\sqrt{1-x^2-y^2}} \]
	      Plug values and simplify:
	      \[ A(s) = \iint_{D}\sqrt{\left(-\frac{x}{\sqrt{1-x^2-y^2}}\right)^2+\left(-\frac{y}{\sqrt{1-x^2-y^2}}\right)^2+1}  \cdot dA \]
	      \[ = \iint_{D}\sqrt{\frac{x^2+y^2}{1-x^2-y^2}+1}  \cdot dA \]
	      \[ = \iint_{D}\sqrt{\frac{1}{1-(x^2+y^2)}}  \cdot dA \]
	      Convert to cylindrical (recall $x^2 + y^2 = r^2$):
	      \[ = \iint_{D}\sqrt{\frac{1}{1-r^2}}  \cdot dA \]
	      Plug values (recall $0 \leq \theta \leq 2\pi $ and $0 \leq r \leq R $, also multiply integral by 2 to account for both hemispheres) and calculate:
	      \[ 2\int_{0}^{1}\int_{\theta=0}^{2\pi} \frac{1}{\sqrt{1-r^2}}\cdot r \cdot dr \cdot d\theta = 2\cdot \int _0^1\frac{2\pi r}{\sqrt{-r^2+1}}dr = 4\pi \]
	      
	\item \textbf{Spherically -- rectangles on surface}
	      
	      Recall the Cartesian $\rightarrow$ spherical conversions:
	      \[ x = \rho\sin(\phi)\cos(\theta)   \]
	      \[ y = \rho\sin(\phi)\sin(\theta)  \]
	      \[ z = \rho\cos(\phi)  \]
	      ...that any rectangle on the surface will be described as:
	      \[ (\rho\cdot\sin(\phi)\cdot d \theta)\cdot(\rho\cdot d \phi) = \rho^2\cdot\sin(\phi)\cdot d\theta\cdot d\phi \]
	      ...and that the coordinate values will range as follows:
	      \[ 0 \leq \phi \leq \pi  \]
	      \[ 0 \leq \theta \leq 2\pi  \]
	      \[ 0 \leq \rho \leq R \]
	      ...and our given value of R is 1 ($\rho = 1$).
	      
	      With this information we can plug and calculate:
	      \[ A =\int_{0}^{\pi}\int_{0}^{2\pi} \rho^2\cdot\sin(\phi)\cdot d\theta\cdot d\phi  \]
	      \[ \int_{0}^{\pi}\int_{0}^{2\pi} (1)^2\cdot\sin(\phi)\cdot d\theta\cdot d\phi\]
	      \[ \int_{0}^{\pi}2\pi \cdot \sin(\phi)\cdot d\phi \]
	      \[ 2\pi \cdot (-\cos(\pi) + \cos(0)) \]
	      \[ 2\pi \cdot (-(-1)+(1)) \]
	      \[ 4\pi\]
	      
	\item \textbf{Stereographically -- vector intersections with the surface}
	      \[ A = \iint_{D} |r_u \times r_v| \cdot dA \]
	      \[x(u,v)=\frac{2u}{\:1^2+u^2+v^2}\]
	      \[y(u,v)=\frac{2v}{\:1^2+u^2+v^2}\]
	      \[z(u,v)=\frac{-1^2+u^2+v^2}{\:1^2+u^2+v^2}\]
	      \[\frac{\partial x}{\partial u} = \frac{2\left(-u^2+v^2+1\right)}{\left(1+u^2+v^2\right)^2}\]
	      \[\frac{\partial y}{\partial u} = -\frac{4vu}{\left(1+u^2+v^2\right)^2}\]
	      \[\frac{\partial z}{\partial u} = \frac{4u}{\left(1+u^2+v^2\right)^2}\]
	      \[\frac{\partial x}{\partial v} = -\frac{4uv}{\left(1+u^2+v^2\right)^2}\]
	      \[\frac{\partial y}{\partial v} = \frac{2\left(1+u^2-v^2\right)}{\left(1+u^2+v^2\right)^2}\]
	      \[\frac{\partial z}{\partial v} = \frac{4v}{\left(1+u^2+v^2\right)^2}\]
	      \[r_u := \frac{\partial x}{\partial u}i + \frac{\partial y}{\partial u}j + \frac{\partial z}{\partial u}k = \left(\frac{2\left(-u^2+v^2+1\right)}{\left(1+u^2+v^2\right)^2}\right)i + \left(-\frac{4vu}{\left(1+u^2+v^2\right)^2}\right)j + \left(\frac{4u}{\left(1+u^2+v^2\right)^2}\right)k\]
	      \[r_v := \frac{\partial x}{\partial v}i + \frac{\partial y}{\partial v}j + \frac{\partial z}{\partial v}k = \left(-\frac{4uv}{\left(1+u^2+v^2\right)^2}\right)i + \left(\frac{2\left(1+u^2-v^2\right)}{\left(1+u^2+v^2\right)^2}\right)j + \left(\frac{4v}{\left(1+u^2+v^2\right)^2}\right)k\]
	      \[ A = \iint_{D} |r_u \times r_v| \cdot dA \]
	      \[\iint_{D} \left|\begin{pmatrix}\frac{8u^3-24uv^2+8u}{\left(u^2+v^2+1\right)^4}&-\frac{8v}{\left(u^2+v^2+1\right)^3}&\frac{4u^4-24u^2v^2+4v^4-4}{\left(u^2+v^2+1\right)^4}\end{pmatrix}\right| dA\]
	      \[\left|\iint_{D} \frac{8u^3-24uv^2+8u}{\left(u^2+v^2+1\right)^4} \cdot dA \cdot i + \iint_{D}\frac{4u^4-24u^2v^2+4v^4-4}{\left(u^2+v^2+1\right)^4} \cdot dA \cdot j + \iint_{D}\frac{8v}{\left(u^2+v^2+1\right)^3}\cdot dA \cdot k \right|\]
	      \[\int_{-\infty}^{\infty}\int_{-\infty}^{\infty} \frac{8u^3-24uv^2+8u}{\left(u^2+v^2+1\right)^4} \cdot du \cdot dv = 0\]
	      \[\int_{-\infty}^{\infty}\int_{-\infty}^{\infty}\frac{4u^4-24u^2v^2+4v^4-4}{\left(u^2+v^2+1\right)^4} \cdot  du \cdot dv = -i\pi\cdot\int_{\infty}^{-\infty}\left(-1\:-\:v^2\right)^{-\frac{5}{2}}\cdot dv = -i\pi \cdot \frac{4i}{3} = \frac{4\pi}{3}\]
	      \[\int_{-\infty}^{\infty}\int_{-\infty}^{\infty}\frac{8v}{\left(u^2+v^2+1\right)^3} \cdot  du \cdot dv = \int_{-\infty}^{\infty}\frac{3\pi v\sqrt{v^2+1}}{v^6+3v^4+3v^2+1} \cdot dv = 0\]
	      \[\left|(0)i + \left(\frac{4\pi}{3}\right)j+(0)k\right|=\sqrt{0^2+\left(\frac{4\pi }{3}\right)^2+0^2}=\frac{4\pi}{3}\]
	      
\end{enumerate}

\section{Volume of a Sphere $x^2 + y^2 + z^2 = 1^2$}
\begin{enumerate}[a.]
	\item \textbf{Cylindrically -- cylindrical "discs" to form sphere}
	      
	      Convert $x^2+y^2$ to $r$ and solve for $r$:
	      \[ r=\sqrt{1-z^2}\]
	      Recall that our cylindrical coordinates range as follows:
	      \[0 \leq r \leq \sqrt{1-z^2}\]
	      \[0 \leq \theta \leq 2\pi \]
	      \[-R \leq z \leq R \]
	      Since the negative  bound turns into two absolute integrals anyway, we can save ourselves the trouble and just range $0 \leq z \leq R$ and multiply the result by 2 to account for both hemispheres.
	      
	      Finally, we plug these values and calculate. To get concentric "discs" stacking from the equator to the pole of the hemisphere, we must integrate with respect to $z$ \textbf{last}:
	      \[ V = 2\cdot \int _0^1\int _0^{2\pi }\int _0^{\sqrt{1-z^2}}rdrd\theta dz \]
	      \[ =2\cdot \int _0^1\int _0^{2\pi }\frac{1-z^2}{2}d\theta dz \]
	      \[=2\cdot \int _0^1\pi \left(1-z^2\right)dz\]
	      \[=2\pi\codt (\int _0^11dz-\int _0^1z^2dz)\]
	      \[=2\pi\left(1-\frac{1}{3}\right)\]
	      \[=\frac{4\pi }{3} \]
	      
	\item \textbf{Spherically -- convert from Cartesian and calculate}
	      
	      Recall spherical Jacobian determinant $ \rho^2\sin(\phi) $ and the Cartesian $\rightarrow$ spherical conversions:
	      \[ x = \rho\sin(\phi)\cos(\theta)   \]
	      \[ y = \rho\sin(\phi)\sin(\theta)  \]
	      \[ z = \rho\cos(\phi)  \]
	      \[ x^2+y^2+z^2 = \rho^2 \]
	      Our volume would take the form
	      \[ V = \iiint_{R}dV \]
	      \[ = \iiint_{R}\rho^2\sin(\phi) \cdot d\rho d\theta d\phi \]
	      Recall that the spherical coordinate values will range as follows:
	      \[ 0 \leq \phi \leq \pi  \]
	      \[ 0 \leq \theta \leq 2\pi  \]
	      \[ 0 \leq \rho \leq R \]
	      Plug these values and calculate:
	      \[ V = \int_{0}^{\pi}\int_{0}^{2\pi}\int_{0}^{(1)}\rho^2\sin(\phi) \cdot d\rho d\theta d\phi \]
	      \[ = \int_{0}^{\pi}\int_{0}^{2\pi}\frac{1}{3}\sin(\phi) d\theta d\phi \]
	      \[ = \frac{1}{3}\int_{0}^{\pi}2\pi\sin(\phi) d\phi \]
	      \[ = \frac{2\pi}{3}\left[-\cos \left(\phi\right)\right]^{\pi }_0\]
	      \[ = \frac{2\pi}{3}(1-\left(-1\right))\]
	      \[ = \frac{4\pi}{3} \]
	      
	\item \textbf{Cylindrically -- convert from Cartesian and calculate}
	      
	      As before, solve for z and convert to $r$ immediately:
	      \[ z=\pm\sqrt{1-(x^2+y^2)} = \pm\sqrt{1-r^2}\]
	      Recall cylindrical Jacobian determinant $ r $ and that cylindrical coordinates range (for one hemisphere, which will multiply later):
	      \[0 \leq \theta \leq 2\pi \]
	      \[0 \leq r \leq R \]
	      \[0 \leq z \leq +\sqrt{1-r^2}\]
	      Since the negative z bound turns into two absolute integrals anyway, we can save ourselves the trouble and just range $0 \leq z \leq +\sqrt{1-r^2}$ and multiply the result by 2 to account for both hemispheres.
	      
	      Finally, we plug these values and calculate. To get concentric cylinders spreading from the center to the outside of the hemisphere, we must integrate with respect to $r$ \textbf{last}:
	      \[ V = \int_{r = 0}^{r = (1)} \int_{\theta = 0}^{\theta = 2\pi} 2\cdot\int_{z = 0}^{z = \sqrt{1-r^2}} r \cdot dz \cdot d\theta \cdot dr \]
	      \[ V = \int_{r = 0}^{r = (1)} \int_{\theta = 0}^{\theta = 2\pi} 2\cdot r \cdot \sqrt{1-r^2} \cdot d\theta \cdot dr \]
	      \[=\int _0^14\pi r\sqrt{-r^2+1} \cdot dr\]
	      \[=4\pi \cdot \int _1^0-u^2 \cdot du\]
	      \[=4\pi \left(-\left(-\int _0^1u^2 \cdot du\right)\right)\]
	      \[=4\pi \left[\frac{u^3}{3}\right]^1_0\]
	      \[=\frac{4\pi }{3} \]
	      
	\item \textbf{Spherically -- order integral to form spherical "shells" emanating from origin}
	      
	      Our volume would take the form
	      \[ V = \iiint_{R}dV \]
	      ...with integration with respect to $\rho$ occurring \textbf{last} in order to get the concentric "shells" effect:
	      \[ = \iiint_{R}\rho^2\sin(\phi) \cdot d\phi d\theta d\rho \]
	      Recall that the spherical coordinate values will range as follows:
	      \[ 0 \leq \phi \leq \pi  \]
	      \[ 0 \leq \theta \leq 2\pi  \]
	      \[ 0 \leq \rho \leq R \]
	      Plug these values and calculate:
	      \[ V = \int_{0}^{(1)}\int_{0}^{2\pi}\int_{0}^{\pi}\rho^2\sin(\phi) \cdot d\phi d\theta d\rho \]
	      \[ \int_{0}^{(1)}\rho^2\int_{0}^{2\pi} \left[-\cos \left(\phi\right)\right]_0^{\pi } d\theta d\rho \]
	      \[ \int_{0}^{(1)}\rho^2\int_{0}^{2\pi} (1-\left(-1\right)) d\theta d\rho \]
	      \[ 2 \cdot \int_{0}^{(1)} 2\pi \cdot \rho^2 d\rho \]
	      \[ 4\pi \cdot \frac{1}{3} \]
	      \[ = \frac{4\pi}{3} \]
	      
	\item \textbf{Stereographically -- vector intersections with the surface}
\end{enumerate}

\section{Gaussian Normal Curve as a Probability Density Function: Prove the volume is 2$\pi$ using polar coordinates}
First, observe that:
\[ \int_{-\infty}^{\infty} e^-\frac{x^2}{2} dx = \int_{-\infty}^{\infty} e^-\frac{y^2}{2} dy\]
Which implies:
\[ \int_{-\infty}^{\infty} e^-\frac{x^2}{2} dx \cdot \int_{-\infty}^{\infty} e^-\frac{y^2}{2} dy \]
\[ = \int_{-\infty}^{\infty} \int_{-\infty}^{\infty} e^-\frac{-(x^2+y^2)}{2} dx dy \]
Now we can can convert to polar coordinates:
\[ \int_{0}^{\infty} \int_{0}^{2\pi} e^{-\frac{r^2}{2}} r \cdot d\theta dr \]
And calculate:
\[ 2\pi \cdot \int_{0}^{\infty} e^{-\frac{r^2}{2}} r \cdot d\theta dr\]
\[ 2\pi \cdot (\lim _{\theta \to \infty \:}\left(-e^{-\frac{\theta ^2}{2}}\right) - \lim _{\theta \to \:0+}\left(-e^{-\frac{\theta ^2}{2}}\right))\]
\[ 2\pi \cdot ((0)-\left(-1\right))\]
\[ 2\pi\]

\section{Sequences and Series}

\begin{enumerate}[a.]
	\item What does it mean for a sequence to converge?
	      
	      When a sequence converges, that means that as you get further and further along the sequence, the terms get closer and closer to a specific limit (usually a real number). For example, to see if the sequence $\frac{1}{n^2}$ converges, we may observe that:
	      \[\lim _{n \to \infty \:}\frac{1}{n^2}=0\]
	      
	\item What does it mean for a series to converge?
	      
	      A series is the sum of a sequence. When a series converges, the sums get closer and closer to a specific limit as we add more and more terms up to infinity. For example, Zeno's dichotomy paradox (halving the distance between objects repeatedly results in them constantly moving but "never" quite reaching each other) is a series:
	      \[D = 1 - \sum_{n=0}^{\infty}\frac{1}{2^n}\]
	      
	\item What does it mean for a series to converge absolutely?
	      
	      A series $\sum a_n$ is called absolutely convergent if $\sum |a_n|$ is convergent. If $\sum a_n$ is convergent and $\sum |a_n|$ is divergent we call the series conditionally convergent.
	      
	\item If a series converges absolutely, does it converge? Prove it.
	      
	      First notice that $|a_n|$ is either $a_n$ or it is $-a_n$ depending on its sign. This means that we can then say, $0 \leq a_n + |a_n| \leq 2|a_n|$
	      
	      Now, since we are assuming that $\sum |a_n| $ is convergent then $\sum |a_n| $ is also convergent since we can just factor the 2 out of the series and 2 times a finite value will still be finite. This however allows us to use the Comparison Test to say that $\sum(a_n + |a_n|)$ is also a convergent series.
	      
	      Finally, we can write, $\sum a_n = \sum(a_n + |a_n|) - \sum |a_n|$ and so $\sum a_n$ is the difference of two convergent series and so is also convergent.
	      
	\item If a series converges, does it converge absolutely? Provide counterexample.
	      
	      Not necessarily, a counterexample would be an alternating series such as
	      \[\sum_{n=0}^{\infty}\frac{(-1)^n}{n}\]
	      which converges but not absolutely, since
	      \[\sum_{n=0}^{\infty}|\frac{(-1)^n}{n}| = \sum_{n=0}^{\infty}\frac{1}{n}\]
	      also known as the harmonic series, which does not converge.
	\item What is a geometric sequence?
	      Geometric sequences follow a pattern of \textbf{multiplying a fixed amount (not zero) from one term to the next}. The number being multiplied each time is constant.
	\item What is a geometric series?
	      A geometric series is the sum of the terms in a geometric sequence. If the sequence has a definite number of terms, the simple formula for the sum is
	      \[s(n) = \frac{a_1(1-r^n)}{1-r}\]
	      This form of the formula is used when the number of terms ($n$), the first term ($a_1$), and the common ratio ($r$) are known.
	\item What criteria guarantee convergence of either a geometric sequence or geometric series?
	      When the absolute of the ratio $r$ of is less than one:
	      \[|r| < 1\]
	\item Prove that the harmonic series diverges, while the series $s(n) = \frac{1}{n^2} $ converges
	      \[s(n) = \frac{1}{n} = 1+\frac{1}{2}+\frac{1}{3}+\frac{1}{4}+\frac{1}{5}+\frac{1}{6}\cdots\]
	      \[\geq 1+\frac{1}{2}+(\frac{1}{4}+\frac{1}{4})+(\frac{1}{8}+\frac{1}{8}+\frac{1}{8}+\frac{1}{8})\cdots\]
	      \[1+\frac{1}{2}+\frac{1}{2}+\frac{1}{2}+\frac{1}{2}+\frac{1}{2}+\frac{1}{2}+\frac{1}{2}\cdots\]
	      With this method, we can use the comparison theorem to prove that:
	      \[\iff \sum_{n=1}{2^k}\frac{1}{n}\geq 1+\frac{k}{2}\]
	      Since $1+\frac{k}{2}$ is clearly infinity, $\sum_{n=1}{\infty}\frac{1}{n}$ must be as well.
	      While it might take more work to calculate what $\frac{1}{n^2}$ converges to exactly, we can prove it does converge by using the integral test:
	      \[\sum_{n=2}^{\infty}\frac{1}{n^2}<\int_{1}^{\infty}\frac{1}{x^2}dx<\sum_{n=1}^{\infty}\frac{1}{n^2}\]
	      The first sum is a rectangular underestimate for the integral, and the second sum is an overestimate. Thus, our series sums to almost the integral, the error being at most the value of the first term.
	      \[\int_{1}^{\infty}\frac{1}{x^2}dx = \lim _{x\to \infty \:}\left(-\frac{1}{x}\right)- \lim _{x\to \:1+}\left(-\frac{1}{x}\right) = (0)-(-1) = 1\]
	      This tells us that the sequence converges and it converges to something between 1 and 2.
	\item Prove with the ratio test that series $e^z$ is absolutely convergent.
	      First, we know that
	      \[e^z=\sum_{n=0}^{\infty}\frac{x^n}{n!}\]
	      Using the ratio test we can show that it converges:
	      \[a_n=\frac{z^n}{n!} \]
	      \[a_{n+1}=\frac{z^{n+1}}{(n+1)!}\]
	      \[\frac{a_{n+1}}{a_n}=\frac{\frac{z^{n+1}}{(n+1)!}}{\frac{z^n}{n!}} = \frac{z^{n+1}n!}{\left(n+1\right)!z^n}=\frac{z^{n-n+1}n!}{\left(n+1\right)!}=\frac{zn!}{\left(n+1\right)!}\]
	      Note the rule:
	      \[\frac{n!}{\left(n+m\right)!}=\frac{1}{\left(n+1\right)\cdot \left(n+2\right)\cdots \left(n+m\right)} \Rightarrow \frac{zn!}{\left(n+1\right)!}=\frac{z}{n+1}\]
	      Finally, we can show that
	      \[\lim_{n\to\infty}|\frac{z}{n+1}|=|z|\cdot(0)=0  \; (\forall \;  z \in \mathbb{R})\]
	\item Prove using the corollary of the root test that the radius of convergence of $e^z$ is infinite.
	      
	      We proved above that $e^z$ converges for all $z\in\mathbb{R}$, however perhaps we can also do so using the root test. The root test tells us that if the following is true, our sequence converges:
	      \[\lim_{n\to\infty}|s(n)|^\frac{1}{n} < 1 \]
	      So we may proceed:
	      \[\lim_{n\to\infty}|s(n)|^\frac{1}{n}\]
	      \[\lim_{n\to\infty}|\frac{z^n}{n!}|^\frac{1}{n}\]
	      \[|x^\infty|^{\frac{1}{\infty}}\cdot\lim_{n\to\infty}|\frac{1}{n!}|^\frac{1}{n}\]
	      \[(1)\cdot\lim_{n\to\infty}\frac{1}{\infty!^\frac{1}{\infty}}\]
	      \[\frac{1}{\infty!^\frac{1}{\infty}}=\frac{1}{\infty}=0\]
	      Thus for all $z\in\mathbb{R}$ we find that $e^z$ runs to 0, which means the radius of convergence is infinite in both directions.
	      
	\item Prove that Euler's formula is true.
	      \[e^{i\theta} = cos(\theta) + i\sin(\theta)\]
	      First, we define e, cos, and sin as power series:
	      \[cos(\theta) = \sum_{n=0}^{\infty} \frac{(-1)^n}{2n!} \cdot (\theta)^2n = 1-\frac{\theta^2}{2!}+\frac{\theta^4}{4!}-\frac{\theta^6}{6!}+\dots\]
	      \[sin(\theta) = \sum_{n=0}^{\infty} \frac{(-1)^{n+1}}{(2n+1)!} \cdot (\theta)^{2n+1} = \theta-\frac{\theta^3}{3!}+\frac{\theta^5}{5!}-\frac{\theta^7}{7!}+\dots\]
	      \[e^{i\theta} = \sum_{n=0}^{\infty} \frac{1}{n!} \cdot (i\theta)^n = 1+\frac{i\theta}{1!}+\frac{i^2\theta^2}{2!}+\frac{i^3\theta^3}{3!}+\frac{i^4\theta^4}{4!}+\frac{i^5\theta^5}{5!}+\frac{i^6\theta^6}{6!}+\frac{i^7\theta^7}{7!}\dots\]
	      \[1+\frac{i\theta}{1!}+\frac{(-1)\theta^2}{2!}+\frac{(-1)i\theta^3}{3!}+\frac{(-1)^2\theta^4}{4!}+\frac{(-1)^2i\theta^5}{5!}+\frac{(-1)^3\theta^6}{6!}+\frac{(-1)^3i\theta^7}{7!}\dots\]
	      \[(1-\frac{\theta^2}{2!}+\frac{\theta^4}{4!}-\frac{\theta^6}{6!}\dots)+\frac{i\theta}{1!}-\frac{i\theta^3}{3!}+\frac{i\theta^5}{5!}-\frac{i\theta^7}{7!}\dots\]
	      \[cos(\theta)+i(\frac{\theta}{1!}-\frac{\theta^3}{3!}+\frac{\theta^5}{5!}-\frac{\theta^7}{7!}\dots)\]
	      \[\cos{\theta}+i\sin{\theta} \]
	      \[\therefore \; e^{i\theta}=\cos{\theta}+i\sin{\theta} \]
	      
\end{enumerate}

\section{Using series to calculate a pesky integral:}
\[\int _{-1 }^{1 \:}e^{-\frac{x^2}{2}}dx\]
Recall that
\[e^z= \sum_{n=0}^{\infty}\frac{x^n}{n!}\]
\[\Rightarrow \int _{-1 }^{1 \:}e^{-\frac{x^2}{2}}dx=\int _{-1 \:}^{1 \:\:}\sum_{n=0}^{\infty}\frac{(-\frac{x^2}{2})^n}{n!} dx \]

Remember: a sequence $s(n)$ converges to a limit
\[L \in\mathbb{R}\iff\forall \; \epsilon > 0  \; \exists \; N \in \mathbb{N} \; s.t. \; |s(n) - L| < \epsilon \; \forall \; n \in \mathbb{N} > N\]
\[ \Rightarrow |\frac{2^{-n}}{n!} - (0)| < \epsilon\]
While solving for $n$ in this case would be difficult, we can try some numbers to see how quickly the error value drops:
\begin{center}
	\begin{tabular}{||c | c | c||}
		\hline
		$n$ & $s(n)$              & $\epsilon_n$    \\ [0.5ex]
		1   & $\frac{2^{-1}}{1!}$ & $\frac{1}{8}$   \\
		2   & $\frac{2^{-2}}{2!}$ & $\frac{1}{48}$  \\
		3   & $\frac{2^{-3}}{3!}$ & $\frac{1}{384}$ \\ [1ex]
		\hline
	\end{tabular}
\end{center}
It appears that in order to reach an error threshold of $2^-8 = \frac{1}{256}$ we will only need 4 terms.
\[\int _{-1 \:}^{1 \:\:}\sum_{n=0}^{3}\frac{(-\frac{x^2}{2})^n}{n!} dx \]
\[\int _{-1}^1\left|\frac{-x^{2\cdot \:0}}{2^0\cdot \:0!}+\frac{-x^{2\cdot \:1}}{2^1\cdot \:1!}+\frac{-x^{2\cdot \:2}}{2^2\cdot \:2!}+\frac{-x^{2\cdot \:3}}{2^3\cdot \:3!}\right|dx=\frac{669}{280}\]

\section{Using a series to calculate $\pi$}
The Generalized binomial formula for $a\in \mathbb R\setminus \{\mathbb N\}$ says
\[(1+x)^a= \sum_{k=0}^{\infty}{a\choose k} x^k\]
Where: ${{a\choose k} =\frac{a(a-1)\cdots(a-k+1)}{k!}}$. In our case $a=\frac{1}{2}$ and $x=(-x^2)$:
\[(1+(-x^2))^{(\frac{1}{2})}= \sum_{k=0}^{\infty}{(\frac{1}{2})\choose k} (-x^2)^k\]
\[1+\left(\frac{1}{2}\right)\left(-x^2\right)+\frac{\left(\frac{1}{2}\right)\left(\left(\frac{1}{2}\right)-1\right)}{2!}\left(-x^2\right)^2+\frac{\left(\frac{1}{2}\right)\left(\left(\frac{1}{2}\right)-1\right)\left(\left(\frac{1}{2}\right)-2\right)}{3!}\left(-x^2\right)^3+\dots\]
This can be simplified and expanded further:
\[1-\frac{x^2}{2}-\frac{x^4}{8}-\frac{x^6}{16}-\frac{5x^8}{128}-\frac{7x^{10}}{256}-\frac{21x^{12}}{1024}\dots\]
We can see above the following rather complicated $s(n)$:
\[s(n) = \frac{\left(2n-3\right)!!}{2^nn!}\]
While solving $\left|s(n) - L\right| < \epsilon$ for $n$ in this case would be difficult, we can try some numbers to see how quickly the error value drops:
\begin{center}
	\begin{tabular}{|| c | c ||}
		\hline
		$n$ & $s(n)$            \\ [0.5ex]
		1   & $\frac{1}{2}$     \\
		2   & $\frac{1}{8}$     \\
		3   & $\frac{1}{16}$    \\
		4   & $\frac{5}{128}$   \\
		5   & $\frac{7}{256}$   \\
		6   & $\frac{21}{1024}$ \\
		7   & $\frac{33}{2048}$ \\
		[1ex]
		\hline
	\end{tabular}
\end{center}
It appears the answer to the question above is "not fast." After trying more numbers I found that an $N$ of $18$ dropped the coefficient within the threshold but I'm not sure how we would determine that quickly.

\section{Using series to calculate $\frac{\pi^2}{6}$ within an error of $2^{-8}$}
Remember: a sequence $s(n)$ converges to a limit
\[L \in\mathbb{R}\iff\forall \; \epsilon > 0  \; \exists \; N \in \mathbb{N} \; s.t. \; |s(n) - L| < \epsilon \; \forall \; n \in \mathbb{N} > N\]
If we solve for n, this should give us the minimum term $N$ with regard to error %\epsilon$.
\[ |\frac{1}{n^2} - (0)| < \epsilon\]
\[n^2>\frac{1}{\epsilon}\]
\[n>\sqrt{\frac{1}{\epsilon}}\]
\[N=\ceil[\bigg]{\sqrt{\frac{1}{\epsilon}}}\]
And finally, we can plug $2^{-8}$ to find $N$:
\[N=\ceil[\bigg]{\sqrt{\frac{1}{(\frac{1}{256})}}}=\sqrt{256}=\sqrt{16^2}=16\]
The \textbf{16th} term in the series should fall within the specified error.
\end{document}