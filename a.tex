\documentclass{article}
\usepackage[utf8]{inputenc}
\usepackage[utf8]{enumerate}

\title{Calculus III Midterm 2 - Solutions}
\date{November 2019}

\begin{document}

\maketitle

\section{Surface Area of a Sphere}
\begin{enumerate}[a.]
	\item Cylindrically -- cylindrical bands on the surface
	    \newline \newline
        Use the gradient form

            $\int\int
                \sqrt{ (\frac{\partial z}{\partial x})^2 + (\frac{\partial z}{\partial y}^2) + 1 } \cdot dA$

            $z = +-\sqrt{1-x^{2}-y^{2}}$

            $\frac{\partial z}{\partial x} = \frac{-x}{\sqrt{1-x^2+y^2}}$

            $\frac{\partial z}{\partial y} = \frac{-y}{\sqrt{1-x^2+y^2}}$

            $0 <= \theta <= 2\pi $

            $0 <= \phi <= \pi $

            $x = \rho\sin(\phi)\cos(\theta) $

            $y = \rho\sin(\phi)\sin(\theta) $

            $z = \rho\cos(\phi) $

            Jacobian$ = \rho^2\sin(\phi) $

            $2\int_{0}^{2\pi}\int_{\theta=0}^{\pi} \frac{1}{\sqrt{1-r^2}}\cdot r dr d\theta$

	\item Spherically -- rectangles on surface

            $\int_{\theta=0}^{\theta=\pi}r\cdot\int_{\phi=0}^{\phi=2\pi} (r sin \theta) d\phi d\theta $

            $= \int_{\theta=0}^{\theta=\pi}r^2 2\pi \sin(\theta)d\theta $

            $=  2\pi r^2 (-\cos(\pi)+\cos(0)) $

            $= 4\pi r^2$

	\item Stereographically -- vector intersections with the surface
\end{enumerate}

\section{Volume of a Sphere}
\begin{enumerate}[a.]
	\item Cylindrically -- cylindrical "shells" to form sphere

        $ 2\int_{r = 0}^{r = R} \int_{\theta = 0}^{\theta = 2\pi} \int_{z = 0}^{z = \sqrt{1-r^2}} r \cdot dz d\theta dr$

        $ = 2 \int_{\theta = 0}^{\theta = 2\pi} \int_{r = 0}^{r = R} r \sqrt{1-r^2} dr d\theta$

        $ = 2 \int_{\theta = 0}^{\theta = 2\pi} \int_{r = 0}^{r = R} -u^2 du d\theta$

        $ = 2 \int_{\theta = 0}^{\theta = 2\pi} \frac{r}{3} d\theta $

        $ = \frac{4}{3}\pi R^3 $

	\item Spherically -- convert from Cartesian and calculate

        $\int_{r=0}^{r=R}\int_{\theta=0}^{\theta=2\pi}\int_{\phi=0}^{\phi=2\pi} dV $

        $ = \int_{r=0}^{r=R}\int_{\theta=0}^{\theta=2\pi} 2\pi (r^2\sin(\theta))  d\theta dr $

        $ = \int_{r=0}^{r=R}4\pi r^2 dr $

        $ = \frac{4}{3}\pi R^3 $

	\item Cylindrically -- convert from Cartesian and calculate

        $8\int_{x = 0}^{x = 1} \int_{y = 0}^{y = \sqrt{1-x^2}} \int_{z = 0}^{z = \sqrt{1-(x^2+y^2)}} dz dy dx$

        $ = 2\int_{r = 0}^{r = R} \int_{\theta = 0}^{\theta = 2\pi} \int_{z = 0}^{z = \sqrt{1-r^2}} r \cdot dz d\theta dr$

        $ = 2 \int_{\theta = 0}^{\theta = 2\pi} \int_{r = 0}^{r = R} \frac{1}{2}\left(-r^2+1\right) dr d\theta$

        $ = 2 \int_{\theta = 0}^{\theta = 2\pi} \frac{1}{3} d\theta $

        $ = \frac{4}{3}\pi R^3 $

	\item Spherically -- order integral to form spherical "shells" emanating from origin

        $\int_{r=0}^{r=R}\int_{\theta=0}^{\theta=2\pi}\int_{\phi=0}^{\phi=2\pi} dV $

        $ = \int_{r=0}^{r=R}\int_{\theta=0}^{\theta=2\pi} 2\pi (r^2\sin(\theta))  d\theta dr $

        $ = \int_{r=0}^{r=R}4\pi r^2 dr $

        $ = \frac{4}{3}\pi R^3 $

	\item Stereographically -- vector intersections with the surface
\end{enumerate}

\section{Gaussian Normal Curve as a Probability Density Function: Prove the volume is 2$\pi$ using polar}

    $\int_{-\infty}^{\infty} e^-\frac{x^2}{2} dx = \int_{-\infty}^{\infty} e^-\frac{y^2}{2} dy $

    $\int_{-\infty}^{\infty} e^-\frac{x^2}{2} dx \cdot \int_{-\infty}^{\infty} e^-\frac{y^2}{2} dy = \int_{-\infty}^{\infty} \int_{-\infty}^{\infty} e^-\frac{-(x^2+y^2)}{2} dx dy $

    $\Rightarrow \int_{-\infty}^{\infty} e^-\frac{x^2}{2} dx = \sqrt{\int_{0}^{\infty} \int_{0}^{2\pi} e^{-\frac{r^2}{2}} r \cdot d\theta dr }$

    $ = \sqrt{\int_{0}^{\infty} e^{-\frac{r^2}{2}} 2\pi r \cdot d\theta dr} $

    $ = \sqrt{2\pi \cdot (-2\pi \cdot \lim _{r\to \infty \:}\left(e^{-\frac{r^2}{2}}\right) - -2\pi e^{-\frac{0^2}{2}})}$

    $ = \sqrt{0-\left(-2\pi \right)}$

    $ = \sqrt{2\pi}$

\section{Sequences and Series}

\begin{enumerate}[a.]
	\item What does it mean for a sequence to converge?
	\item What does it mean for a series to converge?
	\item What does it mean for a series to converge absolutely?
	\item If a series converges absolutely, does it converge? Prove it.
	\item If a series converges, does it converge absolutely? Provide counterexample.
	\item What is a geometric sequence?
	\item What is a geometric series?
	\item What criteria guarantee convergence of either a geometric sequence or geometric series?
	\item Prove that the harmonic series diverges, while the series $s(n) = \frac{1}{n^2} converges to \frac{\pi^2}{6}$
	\item Prove with the ratio test that series $e^z$ is absolutely convergent.
	\item Prove using the corollary of the root test that the radius of convergence of $e^z$ is infinite.
	\item Prove that Euler's formula is true.

	$e^{i\theta} = cos(\theta) + i\sin(\theta)$

	$e^{i\theta} = \sum_{n=0}^{\infty} \frac{1}{n!} \cdot (i\theta)^n$

	$ = 1 + 1(i\theta) + \frac{1}{2!}(\sqrt{-1}\theta)^2 + \frac{1}{3}(\sqrt{-1}\theta)^3 \cdots $

	$ = 1 + i\theta - \frac{1}{2!}\theta^2 - \frac{1}{3}i\theta^3  \cdots$

	$ = 1 - \frac{1}{2!}\theta^2 + (i\theta - \frac{1}{3}i\theta^3)  \cdots$

	$ = 1 - \frac{1}{2!}\theta^2 + i(\theta - \frac{1}{3}\theta^3)  \cdots$

	$cos(\theta) = \sum_{n=0}^{\infty} (-1)^n \cdot \frac{1}{(2n)!} \cdot \theta^{2n}$

	$cos(\theta) = 1 - \frac{1}{2!}\theta^2 \cdots$

	$i\sin(\theta) = \sum_{n=0}^{\infty} (-1)^{n+1} \cdot $

\end{enumerate}

\section{Using series to calculate a pesky integral}

\section{Using series to calculate $\pi$}

\section{Using series to calculate}

$\iint_{a}^{b} x^2 dx$

\end{document}
