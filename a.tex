\documentclass{article}
\usepackage[utf8]{inputenc}
\usepackage{enumerate}
\usepackage{amsmath}
\usepackage{amssymb}
\usepackage{amsfonts}
\usepackage{amstext}
\usepackage{amsthm}

\title{Calculus III Midterm 2 - Solutions}
\date{November 2019}

\begin{document}

\maketitle

\section{Surface Area of a Sphere $x^2 + y^2 + z^2 = 1^2$ }
\begin{enumerate}[a.]
	\item \textbf{Cylindrically -- cylindrical bands on the surface }

            Recall cylindrical Jacobian determinant $ r $ and use the gradient form (assume $R = 1$):
            \[ A = \iint_{D}\sqrt{ (\frac{\partial z}{\partial x})^2 + (\frac{\partial z}{\partial y})^2 + R^2 } \cdot dA \]
            Solve for z:
            \[ z = \pm\sqrt{1-x^{2}-y^{2}} \]
            Compute partials with respect to x then y:
            \[ \frac{\partial z}{\partial x} = \frac{\partial \:}{\partial \:u}\left(\sqrt{u}\right)\frac{\partial \:}{\partial \:x}\left(1-x^2-y^2\right) = \frac{1}{2\sqrt{u}}\left(-2x\right) = -\frac{x}{\sqrt{1-x^2-y^2}} \]
            \[ \frac{\partial z}{\partial y} = \frac{\partial \:}{\partial \:u}\left(\sqrt{u}\right)\frac{\partial \:}{\partial \:y}\left(1-x^2-y^2\right) = \frac{1}{2\sqrt{u}}\left(-2y\right) = -\frac{y}{\sqrt{1-x^2-y^2}} \]
            Plug values and simplify:
            \[ A(s) = \iint_{D}\sqrt{\left(-\frac{x}{\sqrt{1-x^2-y^2}}\right)^2+\left(-\frac{y}{\sqrt{1-x^2-y^2}}\right)^2+1}  \cdot dA \]
            \[ = \iint_{D}\sqrt{\frac{x^2}{1-x^2-y^2}+\frac{y^2}{1-x^2-y^2}+1}  \cdot dA \]
            \[ = \iint_{D}\sqrt{\frac{x^2+y^2}{1-x^2-y^2}+1}  \cdot dA \]
            \[ = \iint_{D}\sqrt{\frac{1}{1-x^2-y^2}}  \cdot dA \]
            \[ = \iint_{D}\sqrt{\frac{1}{1-(x^2+y^2)}}  \cdot dA \]
            Convert to cylindrical (recall $x^2 + y^2 = r^2$):
            \[ = \iint_{D}\sqrt{\frac{1}{1-r^2}}  \cdot dA \]
            Plug values (recall $0 \leq \theta \leq 2\pi $ and $0 \leq r \leq R $, also multiply integral by 2 to account for both hemispheres) and calculate:
            \[ 2\int_{0}^{1}\int_{\theta=0}^{2\pi} \frac{1}{\sqrt{1-r^2}}\cdot r \cdot dr \cdot d\theta = 2\cdot \int _0^1\frac{2\pi r}{\sqrt{-r^2+1}}dr = 4\pi \]

	\item \textbf{Spherically -- rectangles on surface}

            Recall the Cartesian $\rightarrow$ spherical conversions:
            \[ x = \rho\sin(\phi)\cos(\theta)   \]
            \[ y = \rho\sin(\phi)\sin(\theta)  \]
            \[ z = \rho\cos(\phi)  \]
            ...that any rectangle on the surface will be described as:
            \[ (\rho\cdot\sin(\phi)\cdot d \theta)\cdot(\rho\cdot d \phi) = \rho^2\cdot\sin(\phi)\cdot d\theta\cdot d\phi \]
            ...and that the coordinate values will range as follows:
            \[ 0 \leq \phi \leq \pi  \]
            \[ 0 \leq \theta \leq 2\pi  \]
            \[ 0 \leq \rho \leq R \]
            ...and our given value of R is 1 ($\rho = 1$).

            With this information we can plug and calculate:
            \[ A =\int_{0}^{\pi}\int_{0}^{2\pi} \rho^2\cdot\sin(\phi)\cdot d\theta\cdot d\phi  \]
            \[ \int_{0}^{\pi}\int_{0}^{2\pi} (1)^2\cdot\sin(\phi)\cdot d\theta\cdot d\phi\]
            \[ \int_{0}^{\pi}2\pi \cdot \sin(\phi)\cdot d\phi \]
            \[ 2\pi \cdot (-\cos(\pi) + \cos(0)) \]
            \[ 2\pi \cdot (-(-1)+(1)) \]
            \[ 4\pi\]

	\item \textbf{Stereographically -- vector intersections with the surface}
\end{enumerate}

\section{Volume of a Sphere $x^2 + y^2 + z^2 = 1^2$}
\begin{enumerate}[a.]
	\item \textbf{Cylindrically -- cylindrical "discs" to form sphere}

	    Convert $x^2+y^2$ to $r$ and solve for $r$:
	    \[ r=\sqrt{1-z^2}\]
        Recall that our cylindrical coordinates range as follows:
        \[0 \leq r \leq \sqrt{1-z^2}\]
        \[0 \leq \theta \leq 2\pi \]
        \[-R \leq z \leq R \]
        Since the negative  bound turns into two absolute integrals anyway, we can save ourselves the trouble and just range $0 \leq z \leq R$ and multiply the result by 2 to account for both hemispheres.

        Finally, we plug these values and calculate. To get concentric "discs" stacking from the equator to the pole of the hemisphere, we must integrate with respect to $z$ \textbf{last}:
        \[ V = 2\cdot \int _0^1\int _0^{2\pi }\int _0^{\sqrt{1-z^2}}rdrd\theta dz \]
        \[ =2\cdot \int _0^1\int _0^{2\pi }\frac{1-z^2}{2}d\theta dz \]
        \[=2\cdot \int _0^1\pi \left(1-z^2\right)dz\]
        \[=2\pi\codt (\int _0^11dz-\int _0^1z^2dz)\]
        \[=2\pi\left(1\right-\frac{1}{3})\]
        \[=\frac{4\pi }{3} \]

	\item \textbf{Spherically -- convert from Cartesian and calculate}

        Recall spherical Jacobian determinant $ \rho^2\sin(\phi) $ and the Cartesian $\rightarrow$ spherical conversions:
        \[ x = \rho\sin(\phi)\cos(\theta)   \]
        \[ y = \rho\sin(\phi)\sin(\theta)  \]
        \[ z = \rho\cos(\phi)  \]
        \[ x^2+y^2+z^2 = \rho^2 \]
        Our volume would take the form
        \[ V = \iiint_{R}dV \]
        \[ = \iiint_{R}\rho^2\sin(\phi) \cdot d\rho d\theta d\phi \]
        Recall that the spherical coordinate values will range as follows:
        \[ 0 \leq \phi \leq \pi  \]
        \[ 0 \leq \theta \leq 2\pi  \]
        \[ 0 \leq \rho \leq R \]
        Plug these values and calculate:
        \[ V = \int_{0}^{\pi}\int_{0}^{2\pi}\int_{0}^{(1)}\rho^2\sin(\phi) \cdot d\rho d\theta d\phi \]
        \[ = \int_{0}^{\pi}\int_{0}^{2\pi}\frac{1}{3}\sin(\phi) d\theta d\phi \]
        \[ = \frac{1}{3}\int_{0}^{\pi}2\pi\sin(\phi) d\phi \]
        \[ = \frac{2\pi}{3}\left[-\cos \left(\phi\right)\right]^{\pi }_0\]
        \[ = \frac{2\pi}{3}(1-\left(-1\right))\]
        \[ = \frac{4\pi}{3} \]

	\item \textbf{Cylindrically -- convert from Cartesian and calculate}

	    As before, solve for z and convert to $r$ immediately:
	    \[ z=\pm\sqrt{1-(x^2+y^2)} = \pm\sqrt{1-r^2}\]
        Recall cylindrical Jacobian determinant $ r $ and that cylindrical coordinates range (for one hemisphere, which will multiply later):
        \[0 \leq \theta \leq 2\pi \]
        \[0 \leq r \leq R \]
        \[0 \leq z \leq +\sqrt{1-r^2}\]
        Since the negative z bound turns into two absolute integrals anyway, we can save ourselves the trouble and just range $0 \leq z \leq +\sqrt{1-r^2}$ and multiply the result by 2 to account for both hemispheres.

        Finally, we plug these values and calculate. To get concentric cylinders spreading from the center to the outside of the hemisphere, we must integrate with respect to $r$ \textbf{last}:
        \[ V = \int_{r = 0}^{r = (1)} \int_{\theta = 0}^{\theta = 2\pi} 2\cdot\int_{z = 0}^{z = \sqrt{1-r^2}} r \cdot dz \cdot d\theta \cdot dr \]
        \[ V = \int_{r = 0}^{r = (1)} \int_{\theta = 0}^{\theta = 2\pi} 2\cdot r \cdot \sqrt{1-r^2} \cdot d\theta \cdot dr \]
        \[=\int _0^14\pi r\sqrt{-r^2+1} \cdot dr\]
        \[=4\pi \cdot \int _1^0-u^2 \cdot du\]
        \[=4\pi \left(-\left(-\int _0^1u^2 \cdot du\right)\right)\]
        \[=4\pi \left[\frac{u^3}{3}\right]^1_0\]
        \[=\frac{4\pi }{3} \]

	\item \textbf{Spherically -- order integral to form spherical "shells" emanating from origin}

        Our volume would take the form
        \[ V = \iiint_{R}dV \]
        ...with integration with respect to $\rho$ occurring \textbf{last} in order to get the concentric "shells" effect:
        \[ = \iiint_{R}\rho^2\sin(\phi) \cdot d\phi d\theta d\rho \]
        Recall that the spherical coordinate values will range as follows:
        \[ 0 \leq \phi \leq \pi  \]
        \[ 0 \leq \theta \leq 2\pi  \]
        \[ 0 \leq \rho \leq R \]
        Plug these values and calculate:
        \[ V = \int_{0}^{(1)}\int_{0}^{2\pi}\int_{0}^{\pi}\rho^2\sin(\phi) \cdot d\phi d\theta d\rho \]
        \[ \int_{0}^{(1)}\rho^2\int_{0}^{2\pi} \left[-\cos \left(\phi\right)\right]_0^{\pi } d\theta d\rho \]
        \[ \int_{0}^{(1)}\rho^2\int_{0}^{2\pi} (1-\left(-1\right)) d\theta d\rho \]
        \[ 2 \cdot \int_{0}^{(1)} 2\pi \cdot \rho^2 d\rho \]
        \[ 4\pi \cdot \frac{1}{3} \]
        \[ = \frac{4\pi}{3} \]

	\item \textbf{Stereographically -- vector intersections with the surface}
\end{enumerate}

\section{Gaussian Normal Curve as a Probability Density Function: Prove the volume is 2$\pi$ using polar}

    $\int_{-\infty}^{\infty} e^-\frac{x^2}{2} dx = \int_{-\infty}^{\infty} e^-\frac{y^2}{2} dy $

    $\int_{-\infty}^{\infty} e^-\frac{x^2}{2} dx \cdot \int_{-\infty}^{\infty} e^-\frac{y^2}{2} dy = \int_{-\infty}^{\infty} \int_{-\infty}^{\infty} e^-\frac{-(x^2+y^2)}{2} dx dy $

    $\Rightarrow \int_{-\infty}^{\infty} e^-\frac{x^2}{2} dx = \sqrt{\int_{0}^{\infty} \int_{0}^{2\pi} e^{-\frac{r^2}{2}} r \cdot d\theta dr }$

    $ = \sqrt{\int_{0}^{\infty} e^{-\frac{r^2}{2}} 2\pi r \cdot d\theta dr} $

    $ = \sqrt{2\pi \cdot (-2\pi \cdot \lim _{r\to \infty \:}\left(e^{-\frac{r^2}{2}}\right) - -2\pi e^{-\frac{0^2}{2}})}$

    $ = \sqrt{0-\left(-2\pi \right)}$

    $ = \sqrt{2\pi}$

\section{Sequences and Series}

\begin{enumerate}[a.]
	\item What does it mean for a sequence to converge?

	When a sequence converges, that means that as you get further and further along the sequence, the terms get closer and closer to a specific limit (usually a real number). For example, to see if the sequence

	\item What does it mean for a series to converge?

    A series is the sum of a sequence. When a series converges, the sums get closer and closer to a specific limit as we add more and more terms up to infinity. For example, to see if the infinite series

	\item What does it mean for a series to converge absolutely?

    A series $\sum a_n$ is called absolutely convergent if $\sum |a_n|$ is convergent. If $\sum a_n$ is convergent and $\sum |a_n|$ is divergent we call the series conditionally convergent.

	\item If a series converges absolutely, does it converge? Prove it.

    First notice that $|a_n|$ is either $a_n$ or it is $-a_n$ depending on its sign. This means that we can then say, $0 \leq a_n + |a_n| \leq 2|a_n|$

    Now, since we are assuming that $\sum |a_n| $ is convergent then $\sum |a_n| $ is also convergent since we can just factor the 2 out of the series and 2 times a finite value will still be finite. This however allows us to use the Comparison Test to say that $\sum(a_n + |a_n|)$ is also a convergent series.

    Finally, we can write, $\sum a_n = \sum(a_n + |a_n|) - \sum |a_n|$ and so $\sum a_n$ is the difference of two convergent series and so is also convergent.

	\item If a series converges, does it converge absolutely? Provide counterexample.



	\item What is a geometric sequence?
	\item What is a geometric series?
	\item What criteria guarantee convergence of either a geometric sequence or geometric series?
	\item Prove that the harmonic series diverges, while the series $s(n) = \frac{1}{n^2} converges to \frac{\pi^2}{6}$
	\item Prove with the ratio test that series $e^z$ is absolutely convergent.
	\item Prove using the corollary of the root test that the radius of convergence of $e^z$ is infinite.
	\item Prove that Euler's formula is true.

	$e^{i\theta} = cos(\theta) + i\sin(\theta)$

	$e^{i\theta} = \sum_{n=0}^{\infty} \frac{1}{n!} \cdot (i\theta)^n$

	$ = 1 + 1(i\theta) + \frac{1}{2!}(\sqrt{-1}\theta)^2 + \frac{1}{3}(\sqrt{-1}\theta)^3 \cdots $

	$ = 1 + i\theta - \frac{1}{2!}\theta^2 - \frac{1}{3}i\theta^3  \cdots$

	$ = 1 - \frac{1}{2!}\theta^2 + (i\theta - \frac{1}{3}i\theta^3)  \cdots$

	$ = 1 - \frac{1}{2!}\theta^2 + i(\theta - \frac{1}{3}\theta^3)  \cdots$

	$cos(\theta) = \sum_{n=0}^{\infty} (-1)^n \cdot \frac{1}{(2n)!} \cdot \theta^{2n}$

	$cos(\theta) = 1 - \frac{1}{2!}\theta^2 \cdots$

	$i\sin(\theta) = \sum_{n=0}^{\infty} (-1)^{n+1} \cdot $

\end{enumerate}

\section{Using series to calculate a pesky integral: $\int _{-\infty }^{\infty \:}e^{-\frac{x^2}{2}}dx$}

    $= \int _{-\infty \:}^{\infty \:\:}\sum _{n=0}^N\:\frac{1}{n!}\left(\left(-\frac{x^2}{2}\right)-0\right)^n dx$

    $\Rightarrow s(n) = \frac{1}{n!}\left(\left(-\frac{x^2}{2}\right)-0\right)^n = \frac{\left(-\frac{x^2}{2}\right)^n}{n!}$

    Remember: a sequence s(n) converges to a limit
    $L \in\mathbb{R}\iff\forall $
    $\epsilon > 0$
    $ \exists $
    $ N \in \mathbb{N} $
    such that
    $ |s(n) - L| < \epsilon $
    $\forall$
    $ n \in \mathbb{N} > N$

    $\Rightarrow |(\frac{\left(-\frac{x^2}{2}\right)^n}{n!}) - L| < \epsilon$

    We know/assume that: $L = \lim _{n\to \infty }\left(\frac{\left(-\frac{x^2}{2}\right)^n}{n!}\right) = 0$

    $\Rightarrow |\frac{\left(-\frac{x^2}{2}\right)^n}{n!} - (0)| < \epsilon$

    $ = |\frac{\left(-\frac{x^2}{2}\right)^n}{n!}| < \epsilon$

    $ = \ln|\frac{\left(-\frac{x^2}{2}\right)^n}{n!}| < \ln(\epsilon)$

    $ = \ln|\frac{\left(-\frac{x^2}{2}\right)^n}{n!}| < \ln(\epsilon)$

\section{Using series to calculate $\pi$}

\section{Using series to calculate}

$\iint_{a}^{b} x^2 dx$

\end{document}
