\documentclass{article}
\usepackage[utf8]{inputenc}
\usepackage{enumerate}
\usepackage{amsmath}
\usepackage{amssymb}
\usepackage{amsfonts}
\usepackage{amstext}
\usepackage{amsthm}

\title{Calculus III Midterm 2 - Solutions}
\date{November 2019}

\begin{document}

\maketitle

\section{Surface Area of a Sphere}
\begin{enumerate}[a.]
	\item \textbf{Cylindrically -- cylindrical bands on the surface }

            Recall cylindrical Jacobian determinant: $ r $

            Use gradient form (assume $R = 1$):

            \[ A(s) = \iint_{D}\sqrt{ (\frac{\partial z}{\partial x})^2 + (\frac{\partial z}{\partial y})^2 + R^2 } \cdot dA \]

            Solve for z: \[ z = \pm\sqrt{1-x^{2}-y^{2}} \]

            Compute partial with respect to x:

            \[ \frac{\partial z}{\partial x} = \frac{\partial \:}{\partial \:u}\left(\sqrt{u}\right)\frac{\partial \:}{\partial \:x}\left(1-x^2-y^2\right) = \frac{1}{2\sqrt{u}}\left(-2x\right) = -\frac{x}{\sqrt{1-x^2-y^2}} \]

            Compute partial with respect to y:

            \[ \frac{\partial z}{\partial y} = \frac{\partial \:}{\partial \:u}\left(\sqrt{u}\right)\frac{\partial \:}{\partial \:y}\left(1-x^2-y^2\right) = \frac{1}{2\sqrt{u}}\left(-2y\right) = -\frac{y}{\sqrt{1-x^2-y^2}} \]

            Plug values and simplify:

            \[ A(s) = \iint_{D}\sqrt{\left(-\frac{x}{\sqrt{1-x^2-y^2}}\right)^2+\left(-\frac{y}{\sqrt{1-x^2-y^2}}\right)^2+1}  \cdot dA \]
            \[ = \iint_{D}\sqrt{\frac{x^2}{1-x^2-y^2}+\frac{y^2}{1-x^2-y^2}+1}  \cdot dA \]
            \[ = \iint_{D}\sqrt{\frac{x^2+y^2}{1-x^2-y^2}+1}  \cdot dA \]
            \[ = \iint_{D}\sqrt{\frac{1}{1-x^2-y^2}}  \cdot dA \]
            \[ = \iint_{D}\sqrt{\frac{1}{1-(x^2+y^2)}}  \cdot dA \]

            Convert to cylindrical (recall $x^2 + y^2 = r^2$):

            \[ = \iint_{D}\sqrt{\frac{1}{1-(x^2+y^2)}}  \cdot dA \]

            Plug values (recall $0 \leq \theta \leq 2\pi $ and $0 \leq r \leq R $, also multiply integral by 2 to account for both hemispheres) and calculate:

            \[ 2\int_{0}^{1}\int_{\theta=0}^{2\pi} \frac{1}{\sqrt{1-r^2}}\cdot r \cdot dr \cdot d\theta = 2\cdot \int _0^1\frac{2\pi r}{\sqrt{-r^2+1}}dr = 4\pi \]

	\item \textbf{Spherically -- rectangles on surface}

            Recall spherical Jacobian determinant: \[ \rho^2\sin(\phi) \]

            ...the Cartesian $\rightarrow$ spherical conversions:

            \[ x = \rho\sin(\phi)\cos(\theta)   \]
            \[ y = \rho\sin(\phi)\sin(\theta)  \]
            \[ z = \rho\cos(\phi)  \]

            ...that any rectangle on the surface will be described as:

            \[ (\rho\cdot\sin(\theta)\cdot d \phi)\cdot(\rho\cdot d \theta) = \rho^2\cdot\sin(\theta)\cdot d\phi\cdot d\theta \]

            ...and that the coordinate values will range as follows:

            \[ 0 \leq \theta \leq 2\pi  \]
            \[ 0 \leq \phi \leq \pi  \]
            \[ 0 \leq \rho \leq R \]

            With this information we can plug and calculate:

            \[ \int_{\theta=0}^{\theta=2\pi}r\cdot\int_{\phi=0}^{\phi=2\pi} (r sin \theta) d\phi d\theta \]

            $= \int_{\theta=0}^{\theta=\pi}r^2 2\pi \sin(\theta)d\theta $

            $=  2\pi r^2 (-\cos(\pi)+\cos(0)) $

            $= 4\pi r^2$

	\item \textbf{Stereographically -- vector intersections with the surface}
\end{enumerate}

\section{Volume of a Sphere}
\begin{enumerate}[a.]
	\item \textbf{Cylindrically -- cylindrical "shells" to form sphere}

        $ 2\int_{r = 0}^{r = R} \int_{\theta = 0}^{\theta = 2\pi} \int_{z = 0}^{z = \sqrt{1-r^2}} r \cdot dz d\theta dr$

        $ = 2 \int_{\theta = 0}^{\theta = 2\pi} \int_{r = 0}^{r = R} r \sqrt{1-r^2} dr d\theta$

        $ = 2 \int_{\theta = 0}^{\theta = 2\pi} \int_{r = 0}^{r = R} -u^2 du d\theta$

        $ = 2 \int_{\theta = 0}^{\theta = 2\pi} \frac{r}{3} d\theta $

        $ = \frac{4}{3}\pi R^3 $

	\item \textbf{Spherically -- convert from Cartesian and calculate}

        $\int_{r=0}^{r=R}\int_{\theta=0}^{\theta=2\pi}\int_{\phi=0}^{\phi=2\pi} dV $

        $ = \int_{r=0}^{r=R}\int_{\theta=0}^{\theta=2\pi} 2\pi (r^2\sin(\theta))  d\theta dr $

        $ = \int_{r=0}^{r=R}4\pi r^2 dr $

        $ = \frac{4}{3}\pi R^3 $

	\item \textbf{Cylindrically -- convert from Cartesian and calculate}

        $8\int_{x = 0}^{x = 1} \int_{y = 0}^{y = \sqrt{1-x^2}} \int_{z = 0}^{z = \sqrt{1-(x^2+y^2)}} dz dy dx$

        $ = 2\int_{r = 0}^{r = R} \int_{\theta = 0}^{\theta = 2\pi} \int_{z = 0}^{z = \sqrt{1-r^2}} r \cdot dz d\theta dr$

        $ = 2 \int_{\theta = 0}^{\theta = 2\pi} \int_{r = 0}^{r = R} \frac{1}{2}\left(-r^2+1\right) dr d\theta$

        $ = 2 \int_{\theta = 0}^{\theta = 2\pi} \frac{1}{3} d\theta $

        $ = \frac{4}{3}\pi R^3 $

	\item \textbf{Spherically -- order integral to form spherical "shells" emanating from origin}

        $\int_{r=0}^{r=R}\int_{\theta=0}^{\theta=2\pi}\int_{\phi=0}^{\phi=2\pi} dV $

        $ = \int_{r=0}^{r=R}\int_{\theta=0}^{\theta=2\pi} 2\pi (r^2\sin(\theta))  d\theta dr $

        $ = \int_{r=0}^{r=R}4\pi r^2 dr $

        $ = \frac{4}{3}\pi R^3 $

	\item \textbf{Stereographically -- vector intersections with the surface}
\end{enumerate}

\section{Gaussian Normal Curve as a Probability Density Function: Prove the volume is 2$\pi$ using polar}

    $\int_{-\infty}^{\infty} e^-\frac{x^2}{2} dx = \int_{-\infty}^{\infty} e^-\frac{y^2}{2} dy $

    $\int_{-\infty}^{\infty} e^-\frac{x^2}{2} dx \cdot \int_{-\infty}^{\infty} e^-\frac{y^2}{2} dy = \int_{-\infty}^{\infty} \int_{-\infty}^{\infty} e^-\frac{-(x^2+y^2)}{2} dx dy $

    $\Rightarrow \int_{-\infty}^{\infty} e^-\frac{x^2}{2} dx = \sqrt{\int_{0}^{\infty} \int_{0}^{2\pi} e^{-\frac{r^2}{2}} r \cdot d\theta dr }$

    $ = \sqrt{\int_{0}^{\infty} e^{-\frac{r^2}{2}} 2\pi r \cdot d\theta dr} $

    $ = \sqrt{2\pi \cdot (-2\pi \cdot \lim _{r\to \infty \:}\left(e^{-\frac{r^2}{2}}\right) - -2\pi e^{-\frac{0^2}{2}})}$

    $ = \sqrt{0-\left(-2\pi \right)}$

    $ = \sqrt{2\pi}$

\section{Sequences and Series}

\begin{enumerate}[a.]
	\item What does it mean for a sequence to converge?

	When a sequence converges, that means that as you get further and further along the sequence, the terms get closer and closer to a specific limit (usually a real number). For example, to see if the sequence

	\item What does it mean for a series to converge?

    A series is the sum of a sequence. When a series converges, the sums get closer and closer to a specific limit as we add more and more terms up to infinity. For example, to see if the infinite series

	\item What does it mean for a series to converge absolutely?

    A series $\sum a_n$ is called absolutely convergent if $\sum |a_n|$ is convergent. If $\sum a_n$ is convergent and $\sum |a_n|$ is divergent we call the series conditionally convergent.

	\item If a series converges absolutely, does it converge? Prove it.

    First notice that $|a_n|$ is either $a_n$ or it is $-a_n$ depending on its sign. This means that we can then say, $0 \leq a_n + |a_n| \leq 2|a_n|$

    Now, since we are assuming that $\sum |a_n| $ is convergent then $\sum |a_n| $ is also convergent since we can just factor the 2 out of the series and 2 times a finite value will still be finite. This however allows us to use the Comparison Test to say that $\sum(a_n + |a_n|)$ is also a convergent series.

    Finally, we can write, $\sum a_n = \sum(a_n + |a_n|) - \sum |a_n|$ and so $\sum a_n$ is the difference of two convergent series and so is also convergent.

	\item If a series converges, does it converge absolutely? Provide counterexample.



	\item What is a geometric sequence?
	\item What is a geometric series?
	\item What criteria guarantee convergence of either a geometric sequence or geometric series?
	\item Prove that the harmonic series diverges, while the series $s(n) = \frac{1}{n^2} converges to \frac{\pi^2}{6}$
	\item Prove with the ratio test that series $e^z$ is absolutely convergent.
	\item Prove using the corollary of the root test that the radius of convergence of $e^z$ is infinite.
	\item Prove that Euler's formula is true.

	$e^{i\theta} = cos(\theta) + i\sin(\theta)$

	$e^{i\theta} = \sum_{n=0}^{\infty} \frac{1}{n!} \cdot (i\theta)^n$

	$ = 1 + 1(i\theta) + \frac{1}{2!}(\sqrt{-1}\theta)^2 + \frac{1}{3}(\sqrt{-1}\theta)^3 \cdots $

	$ = 1 + i\theta - \frac{1}{2!}\theta^2 - \frac{1}{3}i\theta^3  \cdots$

	$ = 1 - \frac{1}{2!}\theta^2 + (i\theta - \frac{1}{3}i\theta^3)  \cdots$

	$ = 1 - \frac{1}{2!}\theta^2 + i(\theta - \frac{1}{3}\theta^3)  \cdots$

	$cos(\theta) = \sum_{n=0}^{\infty} (-1)^n \cdot \frac{1}{(2n)!} \cdot \theta^{2n}$

	$cos(\theta) = 1 - \frac{1}{2!}\theta^2 \cdots$

	$i\sin(\theta) = \sum_{n=0}^{\infty} (-1)^{n+1} \cdot $

\end{enumerate}

\section{Using series to calculate a pesky integral: $\int _{-\infty }^{\infty \:}e^{-\frac{x^2}{2}}dx$}

    $= \int _{-\infty \:}^{\infty \:\:}\sum _{n=0}^N\:\frac{1}{n!}\left(\left(-\frac{x^2}{2}\right)-0\right)^n dx$

    $\Rightarrow s(n) = \frac{1}{n!}\left(\left(-\frac{x^2}{2}\right)-0\right)^n = \frac{\left(-\frac{x^2}{2}\right)^n}{n!}$

    Remember: a sequence s(n) converges to a limit
    $L \in\mathbb{R}\iff\forall $
    $\epsilon > 0$
    $ \exists $
    $ N \in \mathbb{N} $
    such that
    $ |s(n) - L| < \epsilon $
    $\forall$
    $ n \in \mathbb{N} > N$

    $\Rightarrow |(\frac{\left(-\frac{x^2}{2}\right)^n}{n!}) - L| < \epsilon$

    We know/assume that: $L = \lim _{n\to \infty }\left(\frac{\left(-\frac{x^2}{2}\right)^n}{n!}\right) = 0$

    $\Rightarrow |\frac{\left(-\frac{x^2}{2}\right)^n}{n!} - (0)| < \epsilon$

    $ = |\frac{\left(-\frac{x^2}{2}\right)^n}{n!}| < \epsilon$

    $ = \ln|\frac{\left(-\frac{x^2}{2}\right)^n}{n!}| < \ln(\epsilon)$

    $ = \ln|\frac{\left(-\frac{x^2}{2}\right)^n}{n!}| < \ln(\epsilon)$

\section{Using series to calculate $\pi$}

\section{Using series to calculate}

$\iint_{a}^{b} x^2 dx$

\end{document}
